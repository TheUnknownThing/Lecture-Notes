\documentclass[a4paper]{article}
\usepackage{amsmath, amssymb, amsthm}
\usepackage{enumitem}
\usepackage{framed}
\usepackage{setspace}
\usepackage{indentfirst}
\usepackage[export]{adjustbox}
\usepackage{mathtools}
\usepackage{graphicx}
\usepackage{tabularray}
\usepackage[strict]{changepage}
\usepackage[UTF8]{ctex}
\usepackage[T1]{fontenc}
\usepackage{pgfplots}
\usepackage{etoolbox}
\usepackage{tikz}
\usetikzlibrary{
    patterns,
    arrows,
    positioning,
    calc,
    fadings,
    shapes,
    decorations.markings
}
\usepackage{xcolor}
\usepackage{fancyhdr}
\usepackage{xeCJK}
\usepackage[colorlinks=true, linkcolor=red]{hyperref}
\usepackage{geometry}
\geometry{a4paper, left=25mm, right=25mm, top=25mm, bottom=25mm,}

\newcommand{\e}{\mathrm{e}}
\newcommand{\di}{\mathrm{d}}
\newcommand{\R}{\mathbb{R}}
\newcommand{\N}{\mathbb{N}}
\newcommand{\ep}{\varepsilon}
\newcommand{\st}{,\text{s.t.}}

% 定义 formal 文本框
\definecolor{formalshade}{rgb}{0.95,0.97,1}
\definecolor{formalline}{rgb}{0,0,0.4}
\newenvironment{formal}[1][]{%
\def\FrameCommand{%
\hspace{1pt}%
{\color{formalline}\vrule width 2pt}%
{\color{formalshade}\vrule width 4pt}%
\colorbox{formalshade}%
}%
\MakeFramed{\advance\hsize-\width\FrameRestore}%
\noindent
\hspace{-4.55pt}% disable indenting first paragraph
\begin{adjustwidth}{}{1pt}%
\setlength{\parindent}{0pt}% 移除段落缩进
\vspace{3pt}% 在内容顶部添加垂直空间
\ifx&#1&\else % 检查参数是否为空
\textbf{#1}\par
\vspace{1pt}%
\fi }{%
\vspace{2pt}% 在内容底部添加垂直空间
\end{adjustwidth}\endMakeFramed%
}

% 定义 solution 文本框
\definecolor{solutionshade}{rgb}{1,0.97,0.97}
\definecolor{solutionline}{rgb}{0.7,0.02,0.02}
\newenvironment{solution}[1][]{%
\def\FrameCommand{%
\hspace{1pt}%
{\color{solutionline}\vrule width 2pt}%
{\color{solutionshade}\vrule width 4pt}%
\colorbox{solutionshade}%
}%
\MakeFramed{\advance\hsize-\width\FrameRestore}%
\noindent
\hspace{-4.55pt}% disable indenting first paragraph
\begin{adjustwidth}{}{1pt}%
\setlength{\parindent}{0pt}% 移除段落缩进
\vspace{3pt}% 在内容顶部添加垂直空间
\ifx&#1&\else % 检查参数是否为空
\textbf{#1}\par
\vspace{1pt}%
\fi }{%
\vspace{2pt}% 在内容底部添加垂直空间
\end{adjustwidth}\endMakeFramed%
}

% 定义 problem 文本框
\definecolor{problemshade}{rgb}{0.98,0.96,1}
\definecolor{problemline}{rgb}{0.27,0.2,0.55}
\newenvironment{problem}[1][]{%
\def\FrameCommand{%
\hspace{1pt}%
{\color{problemline}\vrule width 2pt}%
{\color{problemshade}\vrule width 4pt}%
\colorbox{problemshade}%
}%
\MakeFramed{\advance\hsize-\width\FrameRestore}%
\noindent
\hspace{-4.55pt}% disable indenting first paragraph
\begin{adjustwidth}{}{1pt}%
\setlength{\parindent}{0pt}% 移除段落缩进
\vspace{3pt}% 在内容顶部添加垂直空间
\ifx&#1&\else % 检查参数是否为空
\textbf{#1}\par
\vspace{1pt}%
\fi }{%
\vspace{2pt}% 在内容底部添加垂直空间
\end{adjustwidth}\endMakeFramed%
}

% 定义 theorem 文本框
\definecolor{theoremshade}{rgb}{0.98,0.94,0.96}
\definecolor{theoremline}{rgb}{0.8,0.5,0.7}
\newenvironment{theorem}[1][]{%
\def\FrameCommand{%
\hspace{1pt}%
{\color{theoremline}\vrule width 2pt}%
{\color{theoremshade}\vrule width 4pt}%
\colorbox{theoremshade}%
}%
\MakeFramed{\advance\hsize-\width\FrameRestore}%
\noindent
\hspace{-4.55pt}% disable indenting first paragraph
\begin{adjustwidth}{}{1pt}%
\setlength{\parindent}{0pt}% 移除段落缩进
\vspace{3pt}% 在内容顶部添加垂直空间
\ifx&#1&\else % 检查参数是否为空
\textbf{#1}\par
\vspace{1pt}%
\fi }{%
\vspace{2pt}% 在内容底部添加垂直空间
\end{adjustwidth}\endMakeFramed%
}

% 定义 remark 文本框
\newenvironment{remark}{%
\par
\smallskip
\noindent
\itshape Remark. }{%
\smallskip
\hfill$\diamondsuit$ \par}

% 定义 example 文本框
\newenvironment{example}{%
\par
\smallskip
\noindent
\textbf{Example.} }{%
\smallskip
\hfill$\diamondsuit$ \par}

\allowdisplaybreaks
\setlength{\lineskiplimit}{2pt}
\setlength{\lineskip}{\baselineskip-\ccwd}
\setlength{\abovedisplayskip}{4pt}
\setlength{\belowdisplayskip}{4pt}
\setlength{\abovedisplayshortskip}{4pt}
\setlength{\belowdisplayshortskip}{4pt}
\everymath{\displaystyle}
\linespread{1.4}

\title{\huge\textbf{数分(中)课堂笔记}}
\author{TheUnknownThing}

\begin{document}
    \maketitle

    \section{多元函数的性质}

    \subsection{$\mathbb{R}^{n}$ 的度量}

    有内积,就可以诱导(induce)出范数(norm)。通过内积诱导出范数可以这样做:
    \[
        \|x\| = \sqrt{\langle x, x\rangle}
    \]

    有范数,就可以诱导出度量(距离)。度量需要满足的性质是如下三条:
    \begin{enumerate}
        \item $d(x, y) \geq 0$; 且 $d(x, y) = 0$ iff $x = y$

        \item $d(x, y) = d(y, x)$

        \item $d(x, z) \leq d(x, y) + d(y, z)$
    \end{enumerate}

    但是并不一定需要有“范数”,只需要空间有度量即可。

    \begin{formal}
        [Def (metric space)] 设 $X$ 为一集合, $d: X \times X \rightarrow \mathbb{R}
        _{\geq 0}$.

        称 $(X, d)$ 为一 度量空间, 若 $d$ 满足:

        \begin{enumerate}
            \item[(i)] $\forall x, y \in X$, $d(x, y) \geq 0$; 且 $d(x,y) = 0$ iff
                $x = y$

            \item[(ii)] $d(x, y) = d(y, x)$ $\forall x, y \in X$

            \item[(iii)] $d(x, z) \leq d(x, y) + d(y, z)$ $\forall x, y, z \in X$
        \end{enumerate}
    \end{formal}

    有度量,就有邻域。

    \begin{formal}
        [Def (邻域)] 设 $(X, d)$ metric space.

        对 $a \in X$, $r > 0$, 称
        \[
            B_{r}(a) := \{x \in X \mid d(a, x) < r\} \quad (B(a,r))
        \]
        为以 $a$ 为中心, 以 $r$ 为半径的开球, 或称为 $a$ 的 $r$ 邻域.
    \end{formal}

    邻域长成什么样,跟度量有关系,图示两个例子给出不同度量下定义的邻域。

    \begin{example}
        $\mathbb{R}^{n}$, $B_{1}(0)$.

        \begin{itemize}
            \item $n=1$: $(-1, 1)$
                \begin{center}
                    \begin{tikzpicture}
                        \draw[<->] (-1.5,0) -- (1.5,0);
                        \draw[thick] (-1,0) -- (1,0);
                        \draw (0,0) node[below] {$0$};
                        \draw (-1,0) node[below] {$-1$};
                        \draw (1,0) node[below] {$1$};
                    \end{tikzpicture}
                \end{center}

            \item $n=2$:
                \begin{center}
                    \begin{tikzpicture}
                        \draw[->] (-1.5,0) -- (1.5,0);
                        \draw[->] (0,-1.5) -- (0,1.5);
                        \draw (0,0) circle (1cm);
                        \fill[pattern=north east lines, pattern color=gray!50]
                            (0,0)
                            circle
                            (1cm);
                    \end{tikzpicture}
                \end{center}

            \item $n=3$:
                \begin{center}
                    \begin{tikzpicture}
                        \draw (0,0) circle (1.5cm);
                        \draw[dashed] (0,0) ellipse (1.5cm and 0.45cm);
                        \fill[pattern=north east lines, pattern color=gray!50]
                            (0,0)
                            circle
                            (1.5cm);
                    \end{tikzpicture}
                \end{center}
        \end{itemize}

        $(\mathbb{R}^{n}, d_{\infty}), d_{\infty}(x, y) = \max_{1 \leq i \leq n}|
        x^{i}- y^{i}|$

        同样,我们考虑不同维数情况下 $B_{1}(0)$

        \begin{itemize}
            \item $n=1$: $(-1, 1)$
                \begin{center}
                    \begin{tikzpicture}
                        \draw[<->] (-1.5,0) -- (1.5,0);
                        \draw[thick] (-1,0) -- (1,0);
                        \draw (0,0) node[below] {$0$};
                        \draw (-1,0) node[below] {$-1$};
                        \draw (1,0) node[below] {$1$};
                    \end{tikzpicture}
                \end{center}

            \item $n=2$:
                \begin{center}
                    \begin{tikzpicture}
                        \draw[->] (-1.5,0) -- (1.5,0);
                        \draw[->] (0,-1.5) -- (0,1.5);
                        \draw[thick] (-1,-1) rectangle (1,1);
                        \fill[pattern=north east lines, pattern color=gray!50]
                            (-1,-1) rectangle (1,1);
                        \draw (-1,-1) node[below left] {$-1$};
                        \draw (1,-1) node[below right] {$1$};
                        \draw (-1,1) node[above left] {$1$};
                        \draw (1,1) node[above right] {$-1$};
                    \end{tikzpicture}
                \end{center}

            \item $n=3$: 一个边长为 1 的立方体。
        \end{itemize}
    \end{example}

    类似的,我们定义了度量,就可以定义极限。

    \begin{formal}
        [Def] 设 $(X, d)$ metric space.

        设 $\{x_{k}\} \subset X$, $a \in X$, 定义
        \[
            \lim_{k \to \infty}x_{k}= a \overset{\text{def}}{\iff}\forall \varepsilon
            > 0, \exists K, \forall k > K: d(x_{k}, a) < \varepsilon
        \]
    \end{formal}

    我们考虑 $\mathbb{R}^{n}$
    下,使用欧式度量定义的极限。那么我们可以思考:一个“点列”收敛到一个点,其分量是否收敛?不难想象,这个命题其实是平凡的。

    \begin{theorem}[Theorem]
        设 $\{x_{k}\} \subset \mathbb{R}^{n}$, $a \in \mathbb{R}^{n}$.
        \[
            \lim_{k \to \infty}x_{k}= a \iff \forall \, 1 \leq j \leq n, \lim_{k
            \to \infty}x_{k}^{j}= a^{j}
        \]
    \end{theorem}

    \begin{proof}[Proof]
        \[
            |x_{k}^{j}- a^{j}| \leq \|x_{k}- a\| = \sqrt{\sum_{i=1}^{n}(x_{k}^{i}-
            a^{i})^{2}}\leq \sum_{i=1}^{n}|x_{k}^{i}- a^{i}|
        \]
    \end{proof}

    也即,对 $\mathbb{R}^{n}$ 而言,点列收敛,等价于点列中每个分量作为数列收敛。那么刚才我们考虑了有限维,对于无限维呢?

    \begin{remark}
        $\ell^{2}(\mathbb{R}) := \{(x^{1}, x^{2}, \dots) \mid \sqrt{\sum_{i=1}^{\infty}(x^{i})^{2}}
        < +\infty\}$, 设 $\{x_{k}\} \subset \ell^{2}(\mathbb{R}), \forall j \in \mathbb{N}, \lim_{k \to \infty}x_{k}^{j}= a^{j}$

        Question:
        \[
            \lim_{k \to \infty}x_{k}\overset{?}{=}a
        \]
    \end{remark}

    \begin{proof}[Answer]
        答案是否定的,反例:$x_{i}= (0,0,\cdots,1,0,\cdots)$.
    \end{proof}

    由于“点列收敛,等价于点列中每个分量作为数列收敛”,$\mathbb{R}$ 中数列极限的一些性质可以直接搬到
    $\mathbb{R}^{n}$。

    \begin{formal}
        \begin{center}
            \begin{tabular}{c c c}
                \hline
                性质  & $\mathbb{R}$ & $\mathbb{R}^{n}$ \\
                \hline
                唯一性 & $\checkmark$ & $\checkmark$     \\
                有界性 & $\checkmark$ & $\checkmark$     \\
                保号性 & $\checkmark$ & $\times$         \\
                夹逼性 & $\checkmark$ & $\times$         \\
                \hline
            \end{tabular}
        \end{center}

        \bigskip

        定义 $\{x_{k}\}$ 有界: $\exists M > 0$, $\forall k \in \mathbb{N}$,
        $\|x_{k}\| < M$.
    \end{formal}

    \subsection{$\mathbb{R}^{n}$ 的拓扑(开,闭)}

    目标:类比 $(a, b)$ 开区间, $[a, b]$ 闭区间 的 高维情形.

    $(a, b)$: $\forall x \in (a, b)$, $\exists \delta > 0$: $(x - \delta, x + \delta
    ) \subset (a, b)$

    $[a, b]$: 对取极限封闭。 (设 $\{x_{k}\} \subset [a, b]$ 且
    $\lim_{k \to \infty}x_{k}= \xi$, 则 $\xi \in [a, b]$)

    \begin{formal}
        [Def (点与集合的关系)] 设 $E \subset \mathbb{R}^{n}$

        \begin{enumerate}
            \item 称 $x \in \mathbb{R}^{n}$ 为 $E$ 的内点, 若
                $\exists \delta > 0$ s.t. $B_{\delta}(x) \subset E$ (interior
                point)

                记 $E$ 的所有内点构成的集合为 $E^{\circ}$ (也记为 $\operatorname{int}
                (E)$)

                \begin{center}
                    \begin{tikzpicture}
                        \draw (0,0) circle (1.5cm);
                        \node at (0,0) {$E$};
                        \filldraw[fill=gray!20, draw=black]
                            (0.5,0.5)
                            circle
                            (0.2cm);
                        \node at (0.5,0.5) {$x$};
                    \end{tikzpicture}
                \end{center}

            \item 称 $x \in \mathbb{R}^{n}$ 为 $E$ 的外点, 若
                $\exists \delta > 0$ s.t. $B_{\delta}(x) \cap E = \emptyset$ (即
                $x$ 是 $E^{c}$ 的内点)

                记 $E$ 的所有外点构成的集合为 $\operatorname{ext}(E)$ (即 $(E^{c}
                )^{\circ}$)

            \item 称 $x$ 为 $E$ 的边界点, 若
                \[
                    \forall \delta > 0, \quad B_{\delta}(x) \cap E \neq \emptyset
                    \neq B_{\delta}(x) \cap E^{c}
                \]
                记 $E$ 的所有边界点构成的集合为 $\partial E$。
                \[
                    \mathbb{R}^{n}= E^{\circ}\cup (E^{c})^{\circ}\cup \partial E
                    \quad (\text{无交并})
                \]

            \item 称 $x$ 为 $E$ 的聚点 (accumulation point) (也称为极限点 (limiting
                point)), 若
                \[
                    \forall \delta > 0: \#(B_{\delta}(x) \cap E) = \infty
                \]

                记 $E$ 的所有聚点构成的集合为 $E^{\prime}$。

            \item 称 $x$ 为 $E$ 的孤立点 (isolated point) , 若
                \[
                    \exists \delta > 0, \text{s.t. }B_{\delta}(x) \cap E = \{x\}
                \]
        \end{enumerate}
    \end{formal}

    \begin{example}
        \begin{enumerate}
            \item $\mathbb{R}^{n}$, $E = B_{\delta}(a)$ ($\delta > 0$)

                \begin{center}
                    \begin{tikzpicture}
                        \draw (0,0) circle (1.5cm);
                        \node at (-0.2,0) {$a$};
                        \filldraw[fill=gray!20, draw=black]
                            (0.7,0.5)
                            circle
                            (0.1cm);
                        \node at (0.75, 0.5) [right] {$x$};
                        \draw[<->] (0,0) -- (0.7,0.5);
                        \node at (0.35, 0.45) {$r$};
                        \draw[dashed] (0,0) -- (1.5,0);
                        \node at (0.75, -0.5) [above] {$\delta$};
                    \end{tikzpicture}
                \end{center}

                \textbf{Claim:} $E^{\circ}= E$

                \begin{proof}
                    设 $x \in B_{\delta}(a)$. 令 $r = \|a - x\|$, 则 $0 < r < \delta$.

                    取 $\rho = \frac{1}{4}\min\{r, \delta - r\} > 0$. 考虑
                    $B_{\rho}(x)$.

                    $y \in B_{\rho}(x) \implies \|x - y\| < \rho$

                    $\|y - a\| \leq \|y - x\| + \|x - a\| < \rho + r \leq \frac{\delta
                    - r}{4}+ r = \frac{\delta + 3r}{4}< \frac{\delta + 3\delta}{4}
                    = \delta$.

                    因此, $y \in B_{\delta}(a)$. 所以 $B_{\rho}(x) \subset B_{\delta}
                    (a) = E$. 因此 $x \in E^{\circ}$. 所以 $E \subset E^{\circ}$.

                    又因为 $E^{\circ}\subset E$ (根据定义), 所以 $E^{\circ}= E$.
                \end{proof}

                $\operatorname{ext}(E) = \{x \in \mathbb{R}^{n}\mid \|x - a\| > \delta
                \}$

                $\partial E = \{x \in \mathbb{R}^{n}\mid \|x - a\| = \delta\}$

                $E' = \{x \in \mathbb{R}^{n}\mid \|x - a\| \leq \delta\}$

            \item $E = \mathbb{Q}^{n}$

                $E^{\circ}= \emptyset$

                $(E^{c})^{\circ}= \emptyset$

                $E^{\prime}= \mathbb{R}^{n}$
        \end{enumerate}
    \end{example}

    \begin{formal}
        [Def ($\mathbb{R}^{n}$ 的拓扑:开集与闭集)]
        \begin{enumerate}
            \item 称 $G$ 为 $\mathbb{R}^{n}$ 中的开集, 若 $\forall x \in G$, $x$
                是 $G$ 的内点 (即 $G = G^{\circ}$)

            \item 称 $F$ 为 $\mathbb{R}^{n}$ 中的闭集, 若 $\forall x \in F'$, $x
                \in F$ (即 $F \supset F'$)

            \item 称 $\overline{E}:= E \cup E'$ 为 $E$ 的闭包 (closure)
        \end{enumerate}
    \end{formal}

    \begin{remark}
        空集 $\emptyset$ 既是开集也是闭集。
    \end{remark}

    \begin{problem}
        [思考题:证明]
        \begin{enumerate}
            \item $E^{\circ}$ 是 $E$ 中的最大开集 (即若 $G$ 开且 $G \subset E$, 则
                $G \subset E^{\circ}$)

            \item $\overline{E}$ 是包含 $E$ 的最小闭集 (即若 $F$ 闭且
                $F \supset E$, 则 $F \supset \overline{E}$)

            \item 对 $\mathbb{R}^{n}$,
                \[
                    B_{r}(a) = \{x \in \mathbb{R}^{n}\mid d(x, a) < r\}
                \]
                \[
                    \overline{B_r(a)}= \{x \in \mathbb{R}^{n}\mid d(x, a) \leq r
                    \}
                \]

                Q: 设 $(X, d)$ metric space. 是否仍有
                \[
                    \overline{\{x \in X \mid d(x, a) < r\}}\stackrel{?}{=}\{x \in
                    X \mid d(x, a) \leq r\}
                \]
        \end{enumerate}
    \end{problem}

    \begin{theorem}[Theorem]
        $F$ 是 $\mathbb{R}^{n}$ 中的闭集 $\iff$ $F^{c}$ 是 $\mathbb{R}^{n}$ 中的开集.
    \end{theorem}

    \begin{proof}[Proof]
        ($\implies$) 设 $x \in F^{c}$, $x \notin F$, 因 $F$ 闭, $x$ 非 $F$
        的聚点。

        即 $\exists \delta > 0$ s.t. $B_{\delta}(x) \cap F = \emptyset$. 即 $B_{\delta}
        (x) \subset F^{c}$。

        ($\impliedby$) 设 $x \in F'$, 反证, 若 $x \notin F$, 则 $x \in F^{c}$。

        因 $F^{c}$ 开, $\exists \delta > 0$: $B_{\delta}(x) \subset F^{c}$, 与 $x$
        是 $F$ 的聚点矛盾。(因为如果 $x$ 是 $F$ 的聚点,那么所有以 $x$ 为中心的开球都应该与
        $F$ 有非空交集。但是,由于 $F^{c}$ 是开集,我们找到了一个以 $x$ 为中心的开球,它完全位于
        $F^{c}$ 中,与 $F$ 没有任何交集。)
    \end{proof}

    \begin{theorem}[Theorem]
        $\mathbb{R}^{n}$ 中的开、闭集具有以下性质:
        \begin{enumerate}[label=(\roman{*})]
            \item 任意个开集的并仍是开集.

                (设 $\{G_{\alpha}, \alpha \in \Lambda\}$ 是一族开集, 则 $\bigcup_{\alpha
                \in \Lambda}G_{\alpha}$ 是开集.)

            \item 有限个开集的交仍是开集.

            \item 任意个闭集的交仍是闭集.

            \item 有限个闭集的并仍是闭集.
        \end{enumerate}
    \end{theorem}

    \begin{proof}[Proof]
        \begin{enumerate}[label=(\roman{*})]
            \item 设 $x \in \bigcup_{\alpha \in \Lambda}G_{\alpha}$, 则
                $\exists \alpha_{*}\in \Lambda$ \st $x \in G_{*}$.

                因 $G_{*}$ 开, $\exists \delta > 0$ \st $B_{\delta}(x) \subset G_{*}
                \subset \bigcup_{\alpha \in \Lambda}G_{\alpha}$.

            \item 设 $x \in \bigcap_{i=1}^{l}G_{i}$,

                $\forall i = 1, \dots, l$, $x \in G_{i}$. 因 $G_{i}$ 开,
                $\exists \delta_{i}> 0$ \st $B_{\delta_i}(x) \subset G_{i}$.

                取 $\delta := \min\{\delta_{1}, \dots, \delta_{l}\} > 0$, 有 $B_{\delta}
                ( x) \subset G_{i}$, $(\forall i)$,

                $\Rightarrow B_{\delta}(x) \subset \bigcap_{i=1}^{l}G_{i}$.

            \item $\{F_{\alpha}, \alpha \in \Lambda\}$ 一族闭集.

                则
                \[
                    \bigcap_{\alpha \in \Lambda}F_{\alpha}\text{ 闭}\iff \left(\bigcap
                    _{\alpha \in \Lambda}F_{\alpha}\right)^{c}\text{ 开 (de
                    Morgan) }\iff \bigcup_{\alpha \in \Lambda}F_{\alpha}^{c}\text{
                    开 }
                \]
        \end{enumerate}
    \end{proof}

    \begin{formal}
        [Def (拓扑)] 设 $X$ 为一集合, $\tau = 2^{X}$, 称 $(X, \tau)$ 为一拓扑空间,
        若 $\tau$ 满足:
        \begin{enumerate}[label=(\roman{*})]
            \item $\begin{aligned}
                    \emptyset, X \in \tau
                \end{aligned}$

            \item 设 $G_{\alpha}\in \tau$ ($\forall \alpha \in \Lambda$), 则
                $\bigcup_{\alpha \in \Lambda}G_{\alpha}\in \tau$

            \item 设 $G_{1}, G_{2}, \dots, G_{l}\in \tau$, 则
                $\bigcap_{i=1}^{l}G_{i}\in \tau$.
        \end{enumerate}

        将 $\tau$ 中的元素称为 $X$ 中的开集.

        设 $x \in X$, $\exists \tau \ni G \ni x$, 则称 $G$ 是 $x$ 的一个邻域.
    \end{formal}

    \begin{remark}
        $\lim_{k \to \infty}x_{k}= A \stackrel{\text{def}}{\iff}\forall \varepsilon
        > 0, \exists K, \forall k > K$:
        \[
            d(x_{k}, A) < \varepsilon
        \]
        \[
            x_{k}\in B_{\varepsilon}(A)
        \]

        在拓扑空间中 $(X, \tau)$
        \[
            \lim_{k \to \infty}x_{k}= A \stackrel{\text{def}}{\iff}\forall G \in
            \tau \And G \ni A \quad \exists K, \forall k>K: x_{k}\in G
        \]

        $(X, d_{X}) \xrightarrow{f}(Y, d_{Y})$
        \[
            \lim_{x \to x_0}f(x) = A
        \]
    \end{remark}

    \newpage

    \subsection{$\mathbb{R}^{n}$ 的基本定理}

    我们要考虑 $\mathbb{R}^{n}$ 的基本定理之前,我们先来看看我们 $\mathbb{R}$
    中基本定理是如何推导的。

    $\mathbb{R}$ 的构造 $\implies$ 确界存在 $\implies$ 单调有界必收敛 $\implies$
    Bolzano-Weierstrass 定理

    \qquad\qquad\qquad\qquad\qquad\qquad\qquad\qquad\qquad\qquad\qquad\qquad\qquad\qquad\qquad
    $\Downarrow$

    \qquad\qquad\qquad\qquad\qquad\qquad\qquad\qquad\qquad\qquad 闭区间套
    $\Leftarrow$ Cauchy收敛原理

    当然上述的推导的路径只是其中的一条,你可以直接通过B-W推出闭区间套,也可以通过闭区间套推单调有界必收敛……但这个我们现在并不关心,
    我们关心的是,我们能不能把这个推导路径推广到 $\mathbb{R}^{n}$ 中。

    但是和 $\mathbb{R}$ 中不同的是,$\mathbb{R}^{n}$ 无序结构。所以说“单调”就没有了,但是“有界”之类的还是存在的,所以说
    你可以期待 B-W,Cauchy收敛原理,闭区间套这三个定理在 $\mathbb{R}^{n}$ 中的推广。

    \begin{theorem}[Bolzano-Weierstrass]
        $\mathbb{R}^{n}$ 中有界点列 $\{x_{k}\}$ 必有收敛子列.
    \end{theorem}

    \begin{proof}[Proof]
        已知 $\{x_{k}\} \subset \mathbb{R}^{n}$ 中收敛 $\iff \forall j \in \{1, \dots
        , n\}$, $\{x_{k}^{(j)}\}$ ($x_{k}$的第j个分量作为数列) 收敛

        设 $\{x_{k}\}$ 为 $\mathbb{R}^{n}$ 中有界点列.

        则 $\{x_{k}^{(j)}\}_{k=1}^{\infty}$ 为有界数列, $j = 1, \dots, n$.

        令 $j=1, 2, \dots, n$, 利用 $\mathbb{R}$ 的 B-W 定理

        依次取子列即可可得 $\{x_{k_l}\}$ 收敛子列.
    \end{proof}

    \begin{theorem}[Cauchy]
        $\{x_{k}\}$ 在 $\mathbb{R}^{n}$ 中收敛 $\iff$ $\{x_{k}\}$ 是 Cauchy 列.
    \end{theorem}

    \begin{proof}[Proof]
        \begin{itemize}
            \item[$(\Longrightarrow)$] 显然
                \[
                    \begin{aligned}
                        \|x_{k}- x_{l}\| & \le \|x_{k}- A\| + \|A - x_{l}\|
                    \end{aligned}
                \]

            \item[$(\Longleftarrow)$] 有两种方法:
                \begin{enumerate}
                    \item 依照 $\mathbb{R}^{n}$ 的情形, 由 B-W 推出

                    \item $\{x_{n}\}$ Cauchy $\implies$ $\{x_{k}^{(j)}\}$ Cauchy,
                        $j = 1, \dots, n$

                        $\{x_{k}^{(j)}\}$ 收敛, $j = 1, \dots, n$
                \end{enumerate}
        \end{itemize}
    \end{proof}

    \begin{theorem}[Cantor 闭集套]
        设 $\{F_{k}, k \in \N\}$ 是 $\R^{n}$ 中一列非空闭集, 满足
        \begin{itemize}
            \item[(i)] $F_{1}\supset F_{2}\supset F_{3}\supset \dots$

            \item[(ii)] $\lim_{k \to \infty}\mathrm{diam}(F_{k}) = 0$
        \end{itemize}
        其中 $\mathrm{diam}(E) := \sup_{x, y \in E}d(x, y)$.

        则 $\exists! \ \xi \in \bigcap_{k=1}^{\infty}F_{k}$.
    \end{theorem}

    \begin{proof}[Proof]
        $\forall k \in \mathbb{N}_{+}$, 取 $x_{k}\in F_{k}$.

        由闭集套的性质 (i), 有
        \[
            \{x_{k}, x_{k+1}, x_{k+2}, \dots \} \subset F_{k}\ (\forall k \in \mathbb{N}
            _{+})
        \]

        由性质 (ii), $\{x_{k}\}_{k=1}^{\infty}$ 为一个 Cauchy 列, 则
        $\exists \ \xi \in \mathbb{R}^{n}\ \st \lim_{k \to \infty}x_{k}= \xi \ (\forall
        k)$.

        若 $\{x_{k}\}_{k=1}^{\infty}$ 仅有有限个不同的点, 显然成立。

        否则$\{x_{k}\}_{k=1}^{\infty}$ 中有无限个相同的点 $\xi$, 则 $\xi$ 是
        $F_{k}$ 的聚点 $(\forall k)$, $F_{k}$ 闭, 有
        $\xi \in F_{k}\ (\forall k)$.

        故 $\xi \in \bigcap_{k=1}^{\infty}F_{k}$. 唯一性显然.
    \end{proof}

    \begin{remark}
        思考:如何从闭集套推出B-W?

        一维情况下使用二分区间的方式证明,多维情况下如何证明?提示:二维四分,三维八分,$n$
        维 $2^{n}$ 分区间。维数有限非常重要。
    \end{remark}

    接下来我们介绍 $\mathbb{R}^{n}$ 中有界闭集的一个等价刻画。

    \begin{formal}
        [开覆盖] 设 $E \subset \mathbb{R}^{n}$, 若一族开集
        $\{G_{\alpha}, \alpha \in \Lambda\}$ 满足
        \[
            \bigcup_{\alpha \in \Lambda}G_{\alpha}\supset E,
        \]
        则称 $\{G_{\alpha}\}_{\alpha \in \Lambda}$ 为 $E$ 的一个开覆盖.
    \end{formal}

    \begin{formal}
        [紧集] 称 $K \subset \mathbb{R}^{n}$ 为紧集 (Compact Set), 若 $K$
        的任何一个开覆盖, 均存在有限子覆盖.

        「也即:若 $\{G_{\alpha}, \alpha \in \Lambda\}$ 为 $K$ 的一个开覆盖,
        则一定存在
        $G_{\alpha_1}, \dots, G_{\alpha_N}\ \st \ \bigcup_{i=1}^{N}G_{\alpha_i}\supset
        K$.」
    \end{formal}

    \begin{theorem}[Heine-Borel 定理]
        $\mathbb{R}^{n}$ 中
        \[
            K \text{ 紧}\iff K \text{ 有界闭集}
        \]
    \end{theorem}

    \begin{proof}[Proof]
        \begin{itemize}
            \item[$(\Longrightarrow)$] 设 $K$ 是紧集.

                \begin{enumerate}
                    \item 证明 $K$ 有界: 考虑开球族
                        $\{B(0, n) : n \in \mathbb{N}\}$, 显然 $\bigcup_{n=1}^{\infty}
                        B(0, n) = \mathbb{R}^{n}\supset K$. 由于 $K$ 是紧集, 存在有限子覆盖,
                        即存在 $N \in \mathbb{N}$, 使得 $K \subset B(0, N)$.
                        因此, $K$ 是有界的.

                    \item 证明 $K$ 是闭集: 只需证明 $K^{c}$ 是开集. 任取 $x \in K
                        ^{c}$, 对任意 $y \in K$, 由于 $x \ne y$, 存在开球
                        $U_{y}= B(y, r_{y})$ 和 $V_{y}= B(x, r_{y})$ 使得
                        $r_{y}= \frac{1}{2}|x - y|$ 且
                        $U_{y}\cap V_{y}= \emptyset$. 显然 $\{U_{y}: y \in K\}$ 构成
                        $K$ 的一个开覆盖. 由于 $K$ 是紧的, 存在有限子覆盖,
                        即存在 $y_{1}, \dots, y_{m}\in K$, 使得 $K \subset \bigcup
                        _{i=1}^{m}U_{y_i}$. 令 $V = \bigcap_{i=1}^{m}V_{y_i}$, 则
                        $V$ 是 $x$ 的一个开邻域, 且 $V \cap K = \emptyset$. 因此,
                        $V \subset K^{c}$, 这表明 $K^{c}$ 是开集, 从而 $K$
                        是闭集.
                \end{enumerate}

            \item[$(\Longleftarrow)$] 反证法。考虑 $n = 2$ 的情形。

                设 $\{G_{\alpha}\}_{\alpha \in \Lambda}$ 为 $K$ 的一个开覆盖, 但其不存在任何有限开子集可覆盖
                $K$.

                将 $I$ 四等分, 则其中必有一块 (记为 $I_{1}$) 与 $K$ 的交
                $\{G_{\alpha}\}$ 有限覆盖.

                再将 $I_{1}$ 四等分, 则其中必有一块 (记为 $I_{2}$) 与 $K$ 的交 $\{
                G_{\alpha}\}$ 有限覆盖.

                继续可得一列闭矩形 $I_{k}, k=1,2,\dots$

                \begin{enumerate}
                    \item $\forall k$, $I_{k}\cap K$ 不能被
                        $\{G_{\alpha}\}_{\alpha \in \Lambda}$ 的有限个覆盖.

                    \item $\lim_{k \to \infty}\text{diam}(I_{k}\cap K) = 0$.
                \end{enumerate}

                由 Cantor 区间套定理知:
                \[
                    \exists ! \xi \in \bigcap_{k=1}^{\infty}(I_{k}\cap K)
                \]
                \[
                    \xi \in K \subset \bigcup_{\alpha \in \Lambda}G_{\alpha}\Rightarrow
                    \exists G_{*}\st \xi \in G_{*}
                \]

                $G_{*}$ 开 $\Rightarrow \exists \delta > 0 \st B_{\delta}(\xi) \subset
                G_{*}$.

                由于 $\xi \in I_{k}\cap K (\forall k)$

                $\lim_{k \to \infty}\text{diam}(I_{k}\cap K)$, 当 $k$ 充分大时,
                有 $I_{k}\cap K \subset B_{\delta}(\xi)$.

                与 $I_{k}\cap K$ 不能被 $\{G_{\alpha}\}_{\alpha \in \Lambda}$
                有限个覆盖矛盾.
        \end{itemize}
    \end{proof}

    \begin{theorem}
        设 $K \subset \R^{n}$, 则以下三个命题等价:

        (1) $K$ 有界闭.

        (2) $K$ 紧.

        (3) $K$ 中任何序列均有收敛子列, 且其极限 $\in K$. (自列紧)
    \end{theorem}

    \begin{proof}[Proof]
        (1) $\Longleftrightarrow$ (2) 已证.

        (1) $\Longrightarrow$ (3) 显然. (B-W)

        (3) $\Longrightarrow$ (1) 反证法. $K$ 闭是显然的,因为要求聚点都在 $K$
        中。

        下面反证 $K$ 有界. 设 $K$ 无界,则存在 $\{x_{k}\} \subset K$ 满足
        $\|x_{k}\| > k$, 但 $\{x_{k}\}$ 不可能有收敛子列.
    \end{proof}

    \subsection{多元连续映射}

    \begin{formal}
        [多元连续映射] 称 $f: D \subset \R^{n}\longrightarrow \R^{m}$
        \[
            x \longmapsto f(x)
        \]
        为 $n$ 元 $m$ 维向量值函数.

        $D$ 称为 $f$ 的定义域.

        $f(D) := \{f(x) \mid x \in D\}$ 称为 $f$ 的像集.

        $\text{Graph}(f) := \{(x, f(x)) \mid x \in D\}$ 称为 $f$ 的图像.
    \end{formal}

    \begin{formal}
        [多元函数极限] 设 $D \subset \R^{n}$, $x_{0}$ 是 $D$ 的一个聚点. $f: D \setminus
        \{x_{0}\} \to \R^{m}$
        \[
            \lim_{x \to x_0}f(x) = A \stackrel{\text{def}}{\Longleftrightarrow}\forall
            \ep > 0, \; \exists \delta > 0, \; \forall x \in B_{\delta}(x_{0}) \cap
            (D \setminus \{x_{0}\}):
        \]
        \[
            \|f(x) - A\| < \ep
        \]
    \end{formal}

    \begin{remark}
        极限存在意味着:自变量无论以什么样趋近于 $x_{0}$ 时,对应的函数值均趋于 $A$.

        所以, 若 $x$ 沿某条曲线趋近于$x_{0}$时, 极限不存在, 或 $x$
        沿某两条曲线趋近于$x_{0}$时, 得到不同的极限, 则 $f$ 在$x_{0}$处极限不存在.
    \end{remark}

    \begin{problem}
        [E.g.] (1) $f(x, y) = \frac{xy}{x^{2}+ y^{2}}$, $(x, y) \neq (0, 0)$.

        (2) $f(x, y) = \frac{x^{2}y}{x^{4}+ y^{2}}$, $(x, y) \neq (0, 0)$.
    \end{problem}

    \begin{solution}
        (1) 探讨 $\lim_{(x, y) \to (0, 0)}f(x, y)$ 存在与否? 我们沿着 $y = kx$
        趋近于 $(0, 0)$.

        $\lim_{\substack{x \to 0 \\ y = 0}}\frac{xy}{x^{2}+ y^{2}}= 0$.

        $\lim_{\substack{x \to 0 \\ y = kx}}\frac{xkx}{x^{2}+ k^{2}x^{2}}= \frac{k}{1
        + k^{2}}$. 极限不存在.

        (2) $k \neq 0$, 类似的,我们沿着 $y = kx$ 趋近于 $(0, 0)$. $\lim_{\substack{x \to 0 \\ y = kx}}
        \frac{x^{2}kx}{x^{4}+ k^{2}x^{2}}= 0$.

        $\lim_{(x, y) \to (0, 0)}f(x, y)$ 不存在.
    \end{solution}

    与一元函数极限一样,多元函数极限同样拥有唯一性、局部有界性、夹逼性、局部保序性等性质,也同样
    可以进行四则运算。

    也可引入另一种极限概念: 设 $f$ 在 $D$ 上有定义, 若 $\forall y \neq y_{0}$,
    $\lim_{x \to x_0}f(x, y) = \varphi(y)$ 存在, 则可考虑 $\lim_{y \to y_0}\varphi
    (y) = \lim_{y \to y_0}\lim_{x \to x_0}f(x, y )$. (比如你可以先固定 $x$,然后
    让 $y$ 趋近于 $y_{0}$,然后再让 $x$ 趋近于 $x_{0}$) 称为 $f$ 在 $(x_{0}, y_{0}
    )$ 处先 $x$ 后 $y$ 的累次极限.

    \begin{formal}
        [Prop] 设 $f$ 在 $(x_{0}, y_{0})$ 的一个空心邻域上有定义,

        设 $\lim_{(x, y) \to (x_0, y_0)}f(x, y) = A$ 且 $\forall y \neq y_{0}$,
        有 $\lim_{x \to x_0}f(x, y) = \varphi(y)$ 存在,

        则
        \[
            \lim_{y \to y_0}\lim_{x \to x_0}f(x, y) = A.
        \]
    \end{formal}

    讨论完了极限,我们可以继续讨论连续性。

    \begin{formal}
        [连续性] (1) 设 $f: D \to \mathbb{R}^{m}$, $x_{0}\in D$, 称 $f$ 在 $x_{0}$
        处连续, 若
        \[
            \forall \varepsilon > 0, \; \exists \delta > 0, \; \forall x \in B_{\delta}
            (x_{0}) \cap D: \; |f(x) - f(x_{0})| < \varepsilon
        \]

        (2) 若 $\forall x \in D$, $f$ 在$x$处连续, 则称 $f \in C(D, \mathbb{R}^{m}
        )$.

        \vspace{1em}

        如果将定义拓展到度量空间上:设 $(X, d_{X})$, $(Y, d_{Y})$ 为两个度量空间,
        $f: X \to Y$.

        设 $x_{0}\in X$, 称 $f$ 在 $x_{0}$ 处连续, 若
        \[
            \forall \varepsilon > 0, \; \exists \delta > 0, \; \forall x \in X, d
            _{X}(x, x_{0}) < \delta: \; d_{Y}(f(x), f(x_{0})) < \varepsilon.
        \]
    \end{formal}

    \begin{remark}
        若 $x_{0}$ 为 $D$ 的孤立点, 则天然地 $f$ 在 $x$ 处连续; 若 $x_{0}$ 是 $D$
        的聚点, 则 $\lim_{x \to x_0}f(x) = f(x_{0}).$
    \end{remark}

    \begin{formal}
        [Prop] 设 $D \subset \mathbb{R}^{n}$, $x_{0}\in D$, $f: D \to \mathbb{R}^{m}$,

        则 $f$ 在 $x_{0}$ 处连续 $\iff$ $f^{i}: D \to \mathbb{R}$ $(i = 1, \dots,
        m)$ 在 $x_{0}$ 处连续.
    \end{formal}

    \begin{problem}
        [思考题] $f \in C((X, d_{X}), (Y, d_{Y}))$ $\iff$ $\forall$ $Y$ 中的开集
        $V$, 有 $f^{-1}(V)$ 在 $X$ 中是开集. ($f^{-1}(V) = \{x \in X | f(x) \in V
        \}$)
    \end{problem}

    同样,和一元函数一样,多元函数连续性也有四则运算,复合运算等性质。(设 $g: D
    \to \mathbb{R}^{k}$, $f: \Omega \to \mathbb{R}^{m}$, 且 $g(D) \subset \Omega$.
    设 $g$ 在 $x_{0}$ 处连续, $f$ 在 $g(x_{0})$ 处连续, 则 $f \circ g$ 在
    $x_{0}$ 处连续.)

    \subsection{连续映射的整体性质}

    我们这章节需要证明两个东西:

    \begin{formal}
        设 $k \subset \mathbb{R}^{n}$, $f \in C(k, \mathbb{R}^{m})$, 则 $f($紧集$)
        =$紧, $f($连通$)=$连通.
    \end{formal}

    运用这个“高维”的视角再来看一元函数我们证明的结论,我们发现一元情形下是这个的特例。一元时:$[
    a,b]$ 紧,$f([a,b])$ 有界,有最值(相当于就在说 $f([a,b])$ 紧); $[a,b]$
    连续,$f( [a,b])$ 有介值定理(相当于就在说 $f([a,b])$ 连通)。

    \begin{theorem}
        设 $k \subset \mathbb{R}^{n}$, $f \in C(k, \mathbb{R}^{m})$, 则 $f(k)$ 紧.
    \end{theorem}

    我们使用两种方法来证明。两种方法分别对应紧集的两个性质:自列紧和有限开覆盖。

    \begin{proof}[Proof]
        \begin{enumerate}
            \item 设 $\{y_{k}\} \subset f(k)$

                则 $\exists x_{k}\in K$ \st $f(x_{k}) = y_{k}$ $(\forall k)$

                $\{x_{k}\} \subset K$, $K$ 列紧 $\implies$ $\exists$ 子列
                $\{x_{k_l}\} \to \xi \in K$

                $f(k)$ $\xleftarrow{f \text{ 在 } \xi \text{ 处连续}}$ $f(x_{k_l}
                ) = y_{k_l}$

            \item 设 $\{G_{\alpha}, \alpha \in \Lambda\}$ 为 $f(k)$ 的一个开覆盖.

                由 $f$ 的连续性, $f^{-1}(G_{\alpha})$ 在 $k$ 中开
                $(\forall \alpha \in \Lambda)$

                ($f^{-1}($开$)=$开) 且
                $f^{-1}(\bigcup_{\alpha \in \Lambda}G_{\alpha}) = \bigcup_{\alpha
                \in \Lambda}f^{-1}(G_{\alpha}) \supset K$

                $k$ 紧 $\implies$ $f^{-1}(G_{\alpha_1}) \cup \dots \cup f^{-1}(G_{\alpha_l}
                ) \supset K \implies G_{\alpha_1}\cup \dots \cup G_{\alpha_l}\supset
                f(k)$
        \end{enumerate}
    \end{proof}

    \begin{theorem}[Cantor]
        设 $f \in C(k, \mathbb{R}^{m})$, $k \subset \mathbb{R}^{n}$ 紧, 则 $f$ 在
        $k$ 上一致连续.

        即 $\forall \varepsilon > 0$, $\exists \delta > 0$, $\forall x_{1}, x_{2}
        \in k$ 且 $|x_{1}- x_{2}| < \delta$:
        \[
            |f(x_{1}) - f(x_{2})| < \varepsilon
        \]
    \end{theorem}

    \begin{formal}
        [连通] (1) 设 $E \subset \R^{n}$, 设 $\gamma \in C([0,1], \R^{n})$ 且
        $\gamma([0,1]) \subset E$, 则称 $\gamma$ 是 $E$ 中从 $\gamma(0)$ 起至 $\gamma
        (1)$ 终的一条道路.

        (2) 称 $E \subset \R^{n}$ 为 (道路) 连通集, 若 $\forall a, b \in E$, 存在
        $E$ 中一条道路以 $a, b$ 为起止点.
    \end{formal}

    \begin{remark}
        $\mathbb{R}$ 中的道路连通集是区间.
    \end{remark}

    \begin{theorem}
        设 $E \subset \R^{n}$ 连通, $f \in C(E, \R^{m})$, 则 $f(E)$ 连通.
    \end{theorem}

    \begin{proof}
        \begin{tikzpicture}
            % Draw the set E
            \draw[thick]
                (0,0) to[out=45, in=180]
                (2,1) to[out=0, in=90]
                (4,-1) to[out=270, in=0]
                (2,-2) to[out=180, in=315]
                (0,0);
            \node at (2,0.4) {$E$};

            % Draw points x1 and x2 in E
            \filldraw (1,-0.5) circle (1pt) node[anchor=south] {$x_{1}$};
            \filldraw (3,-1.5) circle (1pt) node[anchor=north] {$x_{2}$};

            % Draw a path gamma between x1 and x2
            \draw[->] (1,-0.5) to[out=315, in=135] (3,-1.5);
            \node at (2,-0.7) {$\gamma$};
            \node at (0.5, -1) {$\gamma \in C([0,1], E)$};

            % Draw the mapping f
            \draw[->, thick] (4.5, -1) -- (6.5, -1);
            \node at (5.5, -0.7) {$f$};

            % Draw the set f(E)
            \draw[thick]
                (7,0) to[out=45, in=180]
                (9,1) to[out=0, in=90]
                (11,-1) to[out=270, in=0]
                (9,-2) to[out=180, in=315]
                (7,0);
            \node at (9,0.4) {$f(E)$};

            % Draw points y1 and y2 in f(E)
            \filldraw (8,-0.5)
                circle
                (1pt)
                node[anchor=south] {$y_{1}= f(x_{1})$};
            \filldraw (10,-1.5)
                circle
                (1pt)
                node[anchor=north] {$y_{2}= f(x_{2})$};

            % Draw a path f o gamma between y1 and y2
            \draw[->] (8,-0.5) to[out=315, in=135] (10,-1.5);
            \node at (9,-0.7) {$f \circ \gamma$};
        \end{tikzpicture}
        \[
            \begin{aligned}
                f \circ \gamma \in C([0,1], \R^{n}) \\
                f \circ \gamma([0,1]) \subset f(E)
            \end{aligned}
        \]
        $f \circ \gamma$ 是连接 $y_{1}, y_{2}$ 的一条道路.
    \end{proof}

    \section{多元函数的微分学}

    \subsection{偏导数与微分}

    \begin{formal}
        [偏导数] 设 $D \subset \R^{2}$ 开, $f: D \rightarrow \R$, $(x_{0}, y_{0})
        \in D$, 称
        \[
            \frac{\partial f}{\partial x}(x_{0}, y_{0}) := \lim_{h \to 0}\frac{f(x_{0}+
            h, y_{0}) - f(x_{0}, y_{0})}{h}
        \]
        为 $f$ 在 $(x_{0}, y_{0})$ 处关于 $x$ 的偏导数,也可记为 $f_{x}$.

        $\forall (x, y) \in D$, $f$ 在 $(x, y)$ 处关于 $x$ 的偏导数存在, 则
        \[
            \frac{\partial f}{\partial x}: D \longrightarrow \R
        \]
        \[
            \quad (x, y) \longmapsto \frac{\partial f}{\partial x}
        \]
        称为 $f$ 关于 $x$ 的偏导函数.
    \end{formal}

    \begin{remark}
        (1) $\frac{\partial f}{\partial y}$ 可类似定义
        \[
            \frac{\partial f}{\partial y}(x_{0}, y_{0}) := \lim_{h \to 0}\frac{f(x_{0},
            y_{0}+ h) - f(x_{0}, y_{0})}{h}
        \]

        (2) $f: D \subset \R^{n}\longrightarrow \R$
        \[
            \frac{\partial f}{\partial x^{i}}(X_{0}) := \lim_{h \to 0}\frac{f(x_{0}^{1},
            \cdots, x_{0}^{i-1}, x_{0}^{i}+ h, x_{0}^{i+1}, \cdots, x_{0}^{n}) -
            f(X_{0})}{h}
        \]
    \end{remark}

    \begin{formal}
        [高阶偏导] 设 $f: D \subset \R^{n}\to \R$, 关于 $x^{i}$ 可求偏导, 且 $\frac{\partial
        f}{\partial x^{i}}: D \to \R$ 关于 $x^{j}$ 可继续求偏导, 称其关于
        $x^{j}$ 的偏导数为 $f$ 关于 $x^{i}$, $x^{j}$ 的二阶偏导, 记为
        \[
            \frac{\partial^{2}f}{\partial x^{j}\partial x^{i}}= \frac{\partial}{\partial
            x^{j}}\left( \frac{\partial f}{\partial x^{i}}\right)
        \]
        或
        \[
            f_{x^i x^j}= (f_{x^i})_{x^j}
        \]
    \end{formal}

    \begin{example}
        $f(\vec{x}) = \| \vec{x}- \vec{x}_{0}\|$,
        $\vec{x}\in \R^{n}= \left( \sum_{k=1}^{n}(x^{k}- x_{0}^{k})^{2}\right)^{\frac{1}{2}}$
        \[
            \frac{\partial f}{\partial x^{i}}= \frac{1}{2}\left( \sum_{k=1}^{n}(
            x^{k}- x_{0}^{k})^{2}\right)^{-\frac{1}{2}}\cdot 2(x^{i}- x_{0}^{i})=
            \frac{x^{i}- x_{0}^{i}}{\| \vec{x}- \vec{x}_{0}\|}
        \]
        \[
            \left( \frac{\partial f}{\partial x^{1}}, \cdots, \frac{\partial f}{\partial
            x^{n}}\right) = \frac{\vec{x}- \vec{x}_{0}}{\| \vec{x}- \vec{x}_{0}\|}
        \]
    \end{example}

    \begin{remark}
        $f: D \subset \R^{n}\to \R$
        \[
            \nabla f(x) := \left( \frac{\partial f}{\partial x^{1}}, \cdots, \frac{\partial
            f}{\partial x^{n}}\right)(x)
        \]
        称为 $f$ 的梯度, $\mathrm{grad}f$.
    \end{remark}

    然而,偏导数存在并不意味着函数连续,比如 $f(x, y) = \frac{xy}{x^{2}+ y^{2}}$
    在 $(0, 0)$ 处的偏导数存在,但 $f$ 在 $(0, 0)$ 处极限不存在。我们需要一个更强的性质来保证
    函数的连续性。

    \begin{formal}
        [偏导数有界 $\Rightarrow$ 连续] 设 $f$ 在 $(x_{0}, y_{0})$ 的某个邻域内关于
        $x, y$ 偏导数均存在且有界, 则 $f$ 在 $(x_{0}, y_{0})$ 处连续.
    \end{formal}

    \begin{proof}[Proof]
        \begin{align*}
            |f(x, y) - f(x_{0}, y_{0})| & \le |f(x, y) - f(x_{0}, y)| + |f(x_{0}, y) - f(x_{0}, y_{0})|                                                                                                  \\
                                        & \stackrel{A}{=}\left| \frac{\partial f}{\partial x}(\xi_{1}, y) \right| |x - x_{0}| + \left| \frac{\partial f}{\partial y}(x_{0}, \xi_{2}) \right| |y - y_{0}| \\
                                        & \le M \left( |x - x_{0}| + |y - y_{0}| \right)
        \end{align*}
        $A$: 一元函数微分中值定理. $\xi_{1}\in (x_{0}, x)$,
        $\xi_{2}\in (y_{0}, y)$.
    \end{proof}

    \begin{remark}
        这是一个充分条件,(而且是很强的一个,相当于是一个局部的Lipschitz条件)但不是必要条件,比如
        $\sqrt{x}$ 在 $x=0$ 处的导数不存在,但是在 $x=0$ 处连续。
    \end{remark}

    \begin{formal}
        [方向导数] 设 $f: D \subset \R^{n}\to \R$, $x_{0}\in D$, $\boldsymbol{V}\in
        \R^{n}$, 定义 $f$ 在 $x_{0}$ 处沿 $\boldsymbol{V}$ 的方向导数为 (沿 $\boldsymbol
        {V}$ 正方向逼近)
        \[
            D_{\boldsymbol{V}}f(x_{0}) := \lim_{t \to 0^+}\frac{f(x_{0}+ t\boldsymbol{V})
            - f(x_{0})}{t}
        \]
    \end{formal}

    \begin{remark}
        $\boldsymbol{V}= e_{i}= (0, 0, \underbrace{1}_{i}, 0, \cdots, 0)$ $D_{e_i}
        f(x_{0}) = -D_{-e_i}(x_{0}) \Rightarrow D_{e_i}f(x_{0}) = \frac{\partial
        f}{\partial x^{i}}(x_{0})$.
    \end{remark}

    \begin{example}
        $f = \|x\|$, $x \in \R^{n}$
        \[
            D_{V}f(x) = \lim_{t \to 0^+}\frac{\|x + tV\| - \|x\|}{t}= \frac{\di}{\di
            t}|_{t=0}\|x + tV\|
        \]
        \[
            \frac{\di}{\di t}|_{t=0}\left( \sum_{i=1}^{n}(x^{i}+ tv^{i})^{2}\right
            )^{\frac{1}{2}}= \frac{1}{2}\left( \sum_{i=1}^{n}(x^{i}+ tv^{i})^{2}\right
            )^{-\frac{1}{2}}\cdot 2 \sum_{i=1}^{n}(x^{i}+ tv^{i}) v^{i}|_{t=0}= \frac{x
            \cdot V}{\|x\|}= \nabla f(x) \cdot V
        \]
    \end{example}
\end{document}